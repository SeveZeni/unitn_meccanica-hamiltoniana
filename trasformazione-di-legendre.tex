\section{Trasformazione di Legendre $ t, q^k, \dot{q}^k \longleftrightarrow t, q^k, P_k $}

Siano $ t, q^r, \dot{q}^r $ coordinate naturali sullo spazio-tempo degli stati cinetici, $ \Lagr = \Lagr (t, q^r, \dot{q}^r) $ la funzione Lagrangiana e

\begin{equation}
  \frac{d}{dt} \frac{\partial \Lagr}{\partial \dot{q}^k} - \frac{\partial \Lagr}{\partial q^k} = 0 \qquad k = 1, \dots, n
\end{equation}

le equazioni di Lagrange. \\
Consideriamo la trasformazione di coordinate

\begin{subnumcases}{\label{eq:trasfLegendre}}
  t = t \\
  q^k = q^k & k = 1, \dots, n \\
  P_k = P_k (t, q^r, \dot{q}^r) = \frac{\partial \Lagr (t, q^r, \dot{q}^r)}{\partial \dot{q}^k} \label{eq:trasfLegendre_c}
\end{subnumcases}

detta \textit{trasformazione di Legendre}, che ammette inversa nella forma

\begin{subnumcases} {\label{eq:trasfLegendre_inv}}
  t = t \\
  q^k = q^k & k = 1, \dots, n \\
  \dot{q}^k = \dot{q}^k(t, q^r, P_r)
\end{subnumcases}

in virtù del fatto che

\begin{equation*}
  det\left(\frac{\partial P_k}{\partial \dot{q}^r}\right) = det\left(\frac{\partial}{\partial \dot{q}^r}\frac{\partial \Lagr}{\partial \dot{q}^k}\right) \neq 0.
\end{equation*}

Ricordiamo la definizione dell'integrale generalizzato dell'energia

\begin{equation}
  H = H (t, q^r, \dot{q}^r) = \frac{\partial \Lagr}{\partial \dot{q}^k} \dot{q}^k - \Lagr
\end{equation}

riguardato, ovviamente, come funzione delle coordinate $ t, q^r, \dot{q}^r $. Usando la ($ \ref{eq:trasfLegendre_inv} $) è possibile esprimere $ H $ in funzione delle variabili $ t, q^k, P_k $

\begin{equation} \label{eq:Hamiltoniana}
  \begin{split}
    H &= H(t, q^k, P_k) = H\left(t, q^k, \dot{q}^k(t, q^r, P_r)\right)
    \\
    &= \frac{\partial \Lagr \left(t, q^k, \dot{q}^k(t, q^r, P_r)\right)}{\partial \dot{q}^k} \dot{q}^k - \Lagr \left(t, q^k, \dot{q}^k(t, q^r, P_r)\right)
    \\
    &= P_k \dot{q}^k(t, q^r, P_r) - \Lagr \left(t, q^k, \dot{q}^k(t, q^r, P_r)\right)
  \end{split}
\end{equation}

l'ultima relazione ottenuta usando ($ \ref{eq:trasfLegendre_c} $).\\
Sono vere le identità seguenti, dedotte dalla ($ \ref{eq:Hamiltoniana} $)

\begin{subequations} \label{eq:trasfHamiltoniana}    %FIX label
  \begin{align}
    \frac{\partial H (t, q^r, P_r)}{\partial t} &= P_k \frac{\partial \dot{q}^k}{\partial t} - \frac{\partial \Lagr}{\partial t} - \frac{\partial \Lagr}{\partial \dot{q}^k} \frac{\partial \dot{q}^k}{\partial t} = - \frac{\partial \Lagr}{\partial t} \label{eq:trasfHamiltoniana_a}\\
    \frac{\partial H (t, q^r, P_r)}{\partial q^r} &= P_k \frac{\partial \dot{q}^k}{\partial q^r} - \frac{\partial \Lagr}{\partial q^r} - \frac{\partial \Lagr}{\partial \dot{q}^k} \frac{\partial \dot{q}^k}{\partial q^r} = - \frac{\partial \Lagr}{\partial q^r} \label{eq:trasfHamiltoniana_b}\\
    \frac{\partial H (t, q^r, P_r)}{\partial P_r} &= \frac{\partial P_k}{\partial P_r} \dot{q}^k + P_k \frac{\partial \dot{q}^k}{\partial P_r} - \frac{\partial \Lagr}{\partial \dot{q}^k} \frac{\partial \dot{q}^k}{\partial P_r} = \dot{q}^r \label{eq:trasfHamiltoniana_c}
  \end{align}
\end{subequations}

le semplificazioni derivanti dalla definizione ($ \ref{eq:trasfLegendre_c} $). \\
La ($ \ref{eq:trasfHamiltoniana_c} $) permette, tra l'altro, di scrivere la trasformazione di Legendre diretta e inversa nella forma

\begin{equation} \label{eq:trasfLegendre_inv_compl}
  \begin{cases}
    t = t \\
    q^k = q^k \\
    P_k = \frac{\partial \Lagr (t, q^r, \dot{q}^r)}{\partial \dot{q}^k}
  \end{cases}
  k = 1, \dots, n \quad
  \begin{cases}
    t = t \\
    q^k = q^k\\
    \dot{q}^k = \frac{\partial H(t, q^r, P_r)}{\partial P_k}
  \end{cases}
  k = 1, \dots, n
\end{equation}

Usando ($ \ref{eq:trasfLegendre_c} $), ($ \ref{eq:trasfHamiltoniana_b} $), ($ \ref{eq:trasfHamiltoniana_c} $) si ha una riscrittura delle equazioni di Lagrange nella forma

\begin{equation}
  \begin{cases}
    \frac{d q^k}{d t} = \frac{\partial H(t, q^r, P_r)}{\partial P_k}\\
    \frac{d P_k}{d t} = - \frac{\partial H(t, q^r, P_r)}{\partial q^k}
  \end{cases}
  k = 1, \dots, n 
\end{equation}

che diciamo \textit{Hamiltoniana}.\\
Ricordando che la derivata lungo i moti di $ H $ è

\begin{equation*}
  \left. \frac{d}{d t} \right|_{lungo ~ i ~ moti} H = - \frac{\partial \Lagr (t, q^r, \dot{q}^r)}{\partial t}
\end{equation*}
usando la ($ \ref{eq:trasfHamiltoniana_a} $) si ha che
\begin{equation*}
  \left. \frac{d}{d t} \right|_{lungo ~ i ~ moti} H = \frac{\partial H(t, q^r, P_r)}{\partial t}
\end{equation*}

ovvero che $ H (t, q^r, P_r) $ è un integrale primo se non dipende (esplicitamente) da $ t $.
