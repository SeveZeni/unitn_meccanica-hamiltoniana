\subsection{Una trasformazione che soddisfa la condizione di Lie è simplettica}
\begin{theorem}
Una trasfomazione
\begin{equation*}
 t' = t \qquad q'^i = q'^i (t, q^r, P_r) \qquad P'_i = P'_i (t, q^r, P_r)
\end{equation*}
che verifica la condizione di Lie è \textit{simplettica}, ovvero, detto $ M $ lo jacobiano della trasformazione medesima:

\begin{equation} \label{eq:eq8_condlie}
M = M (t, q^r, P_r) = \left( \begin{array}{cc}
\frac{\partial \textit{q'}}{\partial \textit{q}} & \frac{\partial \textit{q'}}{\partial \textit{P}} \\ 
\frac{\partial \textit{P'}}{\partial \textit{q}} & \frac{\partial \textit{P'}}{\partial \textit{P}}
\end{array} \right)_{2n \times 2n}
\end{equation}

$ M $ soddisfa, per ogni $ t, q^k, P_r $, la condizione (di \textit{simpletticità})
\begin{equation} \label{eq:eq9_condlie}
M^T J M = J
\end{equation}
\end{theorem}

\hphantom{~}

Dimostriamo il teorema nel caso in cui $ F $ dipenda dalle variabili $ t, q^k, q'^k $. Ricordando la condizione di Lie e le sui implicazioni, abbiamo:

\begin{equation*}
P_i dq^i - H dt = P'_i dq'^i - H' dt = \frac{\partial F}{\partial t} dt + \frac{\partial F}{\partial q^i} dq^i + \frac{\partial F}{\partial q'^i} dq'^i
\end{equation*}

% FINE PAGINA 8 - INIZIO PAGINA 9

\begin{subequations} \label{eq:eq10_condlie}
\begin{align}
P_i &= \frac{\partial F (t, q^r, q'^r)}{\partial q^i} \qquad i = 1, \dots , n \label{eq:eq10a_condlie}\\
P'_i &= - \frac{\partial F (t, q^r, q'^r)}{\partial q'^i} \qquad i = 1, \dots , n \label{eq:eq10b_condlie}\\
H' &= H + \frac{\partial F (t, q^r, q'^r)}{\partial t} \label{eq:eq10c_condlie}
\end{align}
\end{subequations}
come già osservato le relazioni (\ref{eq:eq4_condlie}) (o (\ref{eq:eq10_condlie})) definiscono la trasformazione dalle $ t, q^r, P_r $ alle $ t' = t, q'^r, P'_r $ in modo \textit{implicito}, mentre lo jacobiano $ M $ che appare nella condizione ($ \ref{eq:eq9_condlie} $) è definito dalla trasformazione in modo esplicito ($ \Rightarrow $ per calcolare $ M $ dato dalla ($ \ref{eq:eq8_condlie} $) occorre conoscere $ q'^k $ e $ P'_k $ in forma di $ t, q^k, P_k $).
Ovviamente però le ($ \ref{eq:eq10_condlie}a,b $) possono essere lette così:
\begin{subequations} \label{eq:eq11_condlie}
\begin{align}
&P_i = \frac{\partial F}{\partial q^i} \left( t, q^i, q'^i (t, q^k, P_k) \right) \\
&P'_i (t, q^k, P_k) = - \frac{\partial F}{\partial q'^i} \left( t, q^i, q'^i (t, q^k, P_k) \right)
\end{align}
\end{subequations}
in cui le variabili \textit{indipendenti} sono ora $ t, q^k, P_k $. In altre parole, $ q'^i $ e $ P'_i $ che appaiono nelle ($ \ref{eq:eq10a_condlie} $) e ($ \ref{eq:eq10b_condlie} $) sono pensati come funzione di $ t, q^r, P_r $. \\
Evidentemente ora, derivando le ($ \ref{eq:eq11_condlie} a, b $) rispetto alle variabili indipendenti $ t, q^k, P_k $, si ha:

\begin{align*}
& \begin{cases}
0 = \frac{\partial P_i}{\partial q^j} = \frac{\partial^2 F}{\partial q^j \partial q'^i} + \frac{\partial^2 F}{\partial q'^k \partial q^i} \frac{\partial F}{\partial q^j}
\\
\delta_i^j = \frac{\partial P_i}{\partial P_j} = \frac{\partial^2 F}{\partial q'^k \partial q^i} \frac{\partial q'^k}{\partial P_j}
\\
\frac{\partial P'_i}{\partial q_j} = - \frac{\partial^2 F}{\partial q^j \partial q'^i} - \frac{\partial^2 F}{\partial q'^k \partial q'^i} \frac{\partial q'^k}{\partial P_j}
\\
\frac{\partial P'_i}{\partial P_j} = - \frac{\partial^2 F}{\partial q'^k \partial q'^i} \frac{\partial q'^k}{\partial P_j}
\end{cases} & i, j = 1 \dots n
\\
& \begin{cases}
\frac{\partial^2 F}{\partial q^j \partial q^i} + \frac{\partial^2 F}{\partial q^i \partial q'^k} \frac{\partial q'^k}{\partial q^j}
\\
\frac{\partial^2 F}{\partial q^i \partial q'^k} \frac{\partial q'^k}{\partial P_j} = \delta_i^j
\\
\frac{\partial P'_i}{\partial q_j} = - \frac{\partial^2 F}{\partial q^j \partial q'^i} - \frac{\partial^2 F}{\partial q'^k \partial q'^i} \frac{\partial q'^k}{\partial q^j}
\\
\frac{\partial P'_i}{\partial P_j} = - \frac{\partial^2 F}{\partial q'^k \partial q'^i} \frac{\partial q'k}{\partial P_j}
\end{cases} & i, j = 1 \dots n
\end{align*}

Sia $ D_{ik} = \frac{\partial^2 F}{\partial q^i \partial q'^k} $, ovviamente $ det (D_{ik}) \neq 0 $ (vedi formula ($ \ref{eq:eq5_condlie} $) e commenti seguenti).

% FINE PAGINA 9 - INIZIO PAGINA 10

\begin{align*}
& \begin{cases}
\frac{\partial^2 F}{\partial q^j \partial q^i} + D_{ik} \frac{\partial q'^k}{\partial q^j} = 0
\\
D_{ik} \frac{\partial q'^k}{\partial P_j} = \delta_i^j
\\
\frac{\partial P'_i}{\partial q_j} = - D_{ij} - \frac{\partial^2 F}{\partial q'^k \partial q'^i} \frac{\partial q'^k}{\partial q^j}
\\
\frac{\partial P'_i}{\partial P_j} = - \frac{\partial^2 F}{\partial q'^k \partial q'^i} \frac{\partial q'k}{\partial P_j}
\end{cases} & i, j = 1 \dots n
\\
& \begin{cases}
D^{ri} \frac{\partial^2 F}{\partial q^j \partial q^i} + D^{ri} D_{ik} \frac{\partial q'^k}{\partial q^j} = 0 \qquad D^{ri} D_{ik} = \delta_k^r
\\
{D^r}_i D_{ik} \frac{\partial q'^k}{\partial P_j} = \delta_i^j {D^r}_i
\\
\frac{\partial P'_i}{\partial q_j} = - D_{ij} - \frac{\partial^2 F}{\partial q'^k \partial q'^i} \frac{\partial q'^k}{\partial q^j}
\\
\frac{\partial P'_i}{\partial P_j} = - \frac{\partial^2 F}{\partial q'^k \partial q'^i} \frac{\partial q'k}{\partial P_j}
\end{cases} & i, j = 1 \dots n
\\
& \begin{cases}
D^{ri} \frac{\partial^2 F}{\partial q^j \partial q^i} + \frac{\partial q'^r}{\partial q^j} = 0
\\
\frac{\partial q'^r}{\partial P_j} = D^{rj}
\\
\frac{\partial P'_i}{\partial q_j} = - D_{ij} - \frac{\partial^2 F}{\partial q'^k \partial q'^i} \left( -D^{kl} \frac{\partial ^2 F}{\partial q^j \partial q^r} \right)
\\
\frac{\partial P'_i}{\partial P_j} = - \frac{\partial^2 F}{\partial q'^k \partial q'^i} D^{kj}
\end{cases} & i, j = 1 \dots n
\\
& \begin{cases}
\frac{\partial q'^i}{\partial q^j} = - D^{il} \frac{\partial^2 F}{\partial q^j \partial q^l} =- D^{il} \frac{\partial^2 F}{\partial q^l \partial q^j}
\\
\frac{\partial q'^i}{\partial P_j} = D^{ij}
\\
\frac{\partial P'_i}{\partial q_j} = - D_{ij} + \frac{\partial^2 F}{\partial q'^k \partial q'^i}  D^{kl} \frac{\partial ^2 F}{\partial q^j \partial q^l} = - D_{ij} + \frac{\partial^2 F}{\partial q'^i \partial q'^k}  D^{kl} \frac{\partial ^2 F}{\partial q^l \partial q^j}
\\
\frac{\partial P'_i}{\partial P_j} = - \frac{\partial^2 F}{\partial q'^k \partial q'^i} D^{kj} = - \frac{\partial^2 F}{\partial q'^i \partial q'^k} D^{kj}
\end{cases} & i, j = 1 \dots n
\end{align*}

Posto $ D_{ik} = \frac{\partial^2 F}{\partial q^i \partial q'^k} $ , $ G_{ik} = \frac{\partial^2 F}{\partial q'^i \partial q'^k} = G_{ki} $, $ E_{ik} = \frac{\partial^2 F}{\partial q^i \partial q^k} = E_{ki} $ \\

le equazioni viste precedentemente diventano, in forma matriciale:

\begin{equation*}
\frac{\partial q'}{\partial q} = - D^{-1} E \quad \frac{\partial q'}{\partial p} = D^{-1} \quad \frac{\partial P'}{\partial q} = - D^{T} + GD^{-1} E \quad \frac{\partial P'}{\partial p} = - G D^{-1}
\end{equation*}

\begin{equation*}
M = \left( \begin{array}{cc}
\frac{\partial q'}{\partial q} & \frac{\partial q'}{\partial p} \\ 
\frac{\partial P'}{\partial q} & \frac{\partial P'}{\partial p}
\end{array} \right) = \left( \begin{array}{cc}
- D^{-1} E & D^{-1} \\ 
- D^{T} + GD^{-1} E & - G D^{-1}
\end{array} \right) 
\end{equation*}
\begin{equation*}
M^{T} = \left( \begin{array}{cc}
\left( \frac{\partial q'}{\partial q} \right)^{T} & \left( \frac{\partial q'}{\partial p} \right)^{T} \\ 
\left( \frac{\partial P'}{\partial q} \right)^{T} & \left( \frac{\partial P'}{\partial p} \right)^{T}
\end{array} \right) = \left( \begin{array}{cc}
(-D^{-1} E)^{T} & (D^{-1})^{T} \\ 
-D + (GD^{-1}E)^{T}  & (-G D^{-1})^{T}
\end{array} \right)
\end{equation*}

Calcoliamo ora $ M^{T} J M $

\begin{equation*}
\begin{split}
& M^{T} J M =
\\
& = \left( \begin{array}{cc}
-E{(D^{-1})}^T & -D+E{(D^{-1})}^T G \\ 
(D^{-1})^T & -(D^{-1})^T G
\end{array} \right)
\left( \begin{array}{cc}
0 & I \\ 
-I & 0
\end{array} \right)
 \left( \begin{array}{cc}
-D^{-1}E & D^{-1} \\ 
-D^T + GD^{-1} E & -GD^{-1}
\end{array} \right) =
\\
& = \left( \begin{array}{cc}
-E{(D^{-1})}^T & -D+E{(D^{-1})}^T G \\ 
(D^{-1})^T & -(D^{-1})^T G
\end{array} \right)
 \left( \begin{array}{cc}
-D^{-1} + GD^{-1}E & D^{-1} \\ 
D^{-1} E & -D^{-1}
\end{array} \right) =
\\
& = 
 \renewcommand*{\arraystretch}{3}
\left( \begin{array}{c | c}
\begin{aligned}
E{(D^{-1})}^T - E{(D^{-1})}^T GD^{-1} + \\ - D D^{-1} E + E{(D^{-1})}^T GD^{-1} E
\end{aligned}
&
\begin{aligned}
E{(D^{-1})}^T GD^{-1} + DD^{-1} + \\ - E{(D^{-1})}^T GD^{-1}
\end{aligned}
\\
\hline
\begin{aligned}
{(D^{-1})}^T D^T + {(D^{-1})}^T GD^{-1} E + \\ + {(-D^{-1})}^T G D^{-1} E
\end{aligned}
&
\begin{aligned}
-{(D^{-1})}^T G D^{-1} + \\ + (-D^{-1}) G (-D^{-1})
\end{aligned}
\end{array} \right) =
\\
&=
\left( \begin{array}{cc}
0 & I \\ 
-I & 0
\end{array} \right) = J
\end{split}
\end{equation*}

\begin{flushright}
$\square$
\end{flushright}
